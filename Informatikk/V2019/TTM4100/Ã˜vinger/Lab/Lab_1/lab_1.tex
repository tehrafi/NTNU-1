\documentclass[12pt,a4paper]{article}
\usepackage[utf8]{inputenc}
\usepackage{tabularx}
\usepackage{mainpackage}

\begin{document}

\begin{titlepage}
    \centering
    \vspace*{\fill}

    \vspace*{0.5cm}

    \huge
    TTM4100 - Lab 1

    \vspace*{0.5cm}

    \large Sander Lindberg

    \vspace*{\fill}
    \end{titlepage}

	\newpage
	
	\section{Question 1}
		\begin{quote}
			\textit{
				''List 3 different protocols that appear in the protocol column in the unfiltered
				packet-listing window in step 7 above.""			
			}
		\end{quote}
		
		\textbf{Answer: } TCP, DNS, HTTP
		
	\section{Question 2}
		\begin{quote}
			\textit{
				''How long did it take from when the HTTP GET message was sent until the
				HTTP OK reply was received? (By default, the value of the Time column in the
				packet listing window is the amount of time, in seconds, since Wireshark
				tracing began. To display the Time field in time-of-day format, select the
				Wireshark View pull down menu, then select Time Display Format, then select
				Time-of-day.)''			
			}
		\end{quote}
	
		\textbf{Answer: } 
		\begin{figure}[hb!]
			\centering
			\includegraphics[width=\linewidth]{lab_1_oppgave_2.png}
			\caption{Snippet of Wireshark trace}
			\label{fig:oppgave2}
		\end{figure}
		
		The GET message was sent \textbf{12:06:14.843692} and the OK message was recieved \textbf{12:06:14.991033}, so it took about \textbf{0.147341} 			seconds.
		
		\section{Question 3}
			\begin{quote}
				\textit{
					''What is the Internet address of the gaia.cs.umass.edu (also known as
					www.net.cs.umass.edu)? What is the Internet address of your computer?''				
				}
			\end{quote}
			
			\textbf{Answer: } From figure \ref{fig:oppgave2}, we see that the Internet address of \textbf{gaia.cs.umass.edu} is \textbf{128.119.245.12} (The 							destination in the GET message) and the Internet address for my computer is \textbf{192.168.1.17} (the source address in the GET message).
			
			\section{Question 4}
				\begin{quote}
					\textit{
						''Take screenshots of the two HTTP messages (GET and OK) referred to in
						question 2 above. The screenshots should include the packet-header window for
						these messages and the packet list window (see the beginning of this tutorial for
						the descriptions of the different windows).''					
					}
				\end{quote}
				
				\textbf{Answer: } 
				\begin{figure}[hb!]
					\centering
					\includegraphics[width=\linewidth]{get.png}
					\caption{Get message}
					\label{fig:get_message}
				\end{figure}
				
				\begin{figure}[hb!]
					\centering
					\includegraphics[width=\linewidth]{OK.png}
					\caption{OK message}
					\label{fig:OK_message}
				\end{figure}
\end{document}