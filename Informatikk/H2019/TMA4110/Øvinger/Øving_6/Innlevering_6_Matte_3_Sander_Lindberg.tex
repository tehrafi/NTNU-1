\documentclass[11pt, a4paper, norsk]{NTNUoving}
\usepackage[utf8]{inputenc}
\usepackage[T1]{fontenc}
\makeatletter

\renewcommand*\env@matrix[1][*\c@MaxMatrixCols c]{%
  \hskip -\arraycolsep
  \let\@ifnextchar\new@ifnextchar
  \array{#1}}
\makeatother

\ovingnr{6}    % Nummer på innlevering
\semester{Høsten 2019}
\fag{TMA 4110}
\institutt{Institutt for matematiske fag}

\begin{document}

   \begin{oppgave}
       Bruk minste kvadraters metode på det overbestemte systemet
       \begin{punkt}
           $\begin{bmatrix}[cc|c]
               2 & 1 & -1 \\
               -3 & 1 & -2 \\
               -1 & 1 & 1
           \end{bmatrix}$
           
           Må løse \[A^{*}A\underline{x} = A^{*}\underline{b}\] Jeg har \[A=\begin{bmatrix}
               2 & 1 \\
               -3 & 1 \\
               -1 & 1
           \end{bmatrix}, \; A^{*} = \begin{bmatrix}
               2 & -3 & -1 \\
               1 & 1 & 1
           \end{bmatrix}, \; b = \begin{bmatrix}
               -1 \\
               -2 \\
               1
           \end{bmatrix} \]

           Finner først $A^{*}A\underline{x}$:

           \begin{align*}
               A^{*}A\underline{x} &= \begin{bmatrix}
                   2 & -3 & 1 \\
                   1 & 1 & 1
               \end{bmatrix}\begin{bmatrix}
                   2 & 1 \\
                   -3 & 1 \\
                   -1 & 1
               \end{bmatrix}\underline{x} 
               \\
                                   &= \begin{bmatrix}
                                       14 & -2 \\
                                       -2 & 1
                                   \end{bmatrix}\underline{x}
           \end{align*}

           Finner så $A^{*}\underline{b}$

           \begin{align*}
               A^{*}\underline{b} = \begin{bmatrix}
                   2 & -3 & -1 \\
                   1 & 1 & 1
               \end{bmatrix}\begin{bmatrix}
                    -1 \\
                    -2 \\
                    1
               \end{bmatrix} = \begin{bmatrix}
                   3 \\
                   -2
               \end{bmatrix}
           \end{align*}
           
           Løser nå $A^{*}A\underline{x} = A^{*}\underline{b}$
           \begin{align*}
               \begin{bmatrix}[cc|c]
                   14 & -2 & 3 \\
                   -2 & 1 & -2
                   \end{bmatrix} &\sim \begin{bmatrix}[cc|c]
               14 & -2 & 3 \\
               1 & -\frac{1}{2} & 1
               \end{bmatrix}
               \\
               &\sim \begin{bmatrix}[cc|c]
                   0 & 5 & -11 \\
                   1 & -\frac{1}{2} & 1
               \end{bmatrix}
               \\
               &\sim \begin{bmatrix}[cc|c]
                   0 & 1 & -\frac{11}{5}\\
                   1 & 0 & -\frac{1}{10}
               \end{bmatrix}
           \end{align*}
           
           Hvor jeg fant $\hat{x} = \begin{bmatrix}
               -\frac{11}{5} \\
               -\frac{1}{10}
           \end{bmatrix}$

           Jeg må nå finne $A\hat{x}$:
           \begin{align*}
               \begin{bmatrix}
                   2 & 1 \\
                   -3 & 1 \\
                   -1 & 1
               \end{bmatrix}\begin{bmatrix}
                   -\frac{11}{5} \\
                   -\frac{1}{10}
               \end{bmatrix} = \begin{bmatrix}
                   -\frac{9}{2} \\
                   \frac{33}{5} \\
                   \frac{21}{10}
               \end{bmatrix}
           \end{align*}
           
       \end{punkt}
        \begin{punkt}
            $\begin{bmatrix}[ccc|c]
                0 & 1 & 1 & 1-i \\
                i & i & -1 & 1+i \\
                0 & i & 0 & i \\
                0 & i & 1 & 1
            \end{bmatrix}$

            Finner igjen $A^{*}A\underline{x} = A^{*}\underline{b}$.
            
            Jeg har \[ A = \begin{bmatrix}
                0 & 1 & 1 \\
                i & i & -1 \\
                0 & i & 1
            \end{bmatrix}, \; A^{*} = \begin{bmatrix}
                0 & -i & 0 & 0 \\
                1 & -i & -i & -i \\
                1 & -1 & 0 & 1
            \end{bmatrix}, \; \underline{b} = \begin{bmatrix}
                1-i \\
                1+i \\
                i \\
                1
            \end{bmatrix}\]

            Finner $A^{*}A\underline{x}$:
            \begin{align*}
                \begin{bmatrix}
                0 & -i & 0 & 0 \\
                1 & -i & -i & -i \\
                1 & -1 & 0 & 1
                \end{bmatrix} \begin{bmatrix}
                0 & 1 & 1 \\
                i & i & -1 \\
                0 & i & 0 \\
                0 & i & 1
            \end{bmatrix} \underline{x} &= \begin{bmatrix}
            1 & 1 & 1 \\
            1 & 4 & 1 \\
            -i & 1 & 3
            \end{bmatrix}\begin{bmatrix}
                x_1 \\
                x_2 \\
                x_3
            \end{bmatrix}
            \end{align*}
                
            Finner $A^{*}\underline{b}$
            \begin{align*}
                \begin{bmatrix}
                    0 & -i & 0 & 0 \\
                    1 & -i & -i & -i \\
                    1 & -1 & 0 & 1
                \end{bmatrix}\begin{bmatrix}
                    1-i \\
                    1+i \\
                    i \\
                    1
                \end{bmatrix} &= \begin{bmatrix}
                    1-i \\
                    3-3i \\
                    1-2i
                \end{bmatrix}
            \end{align*}
            Løser nå $A^{*}A\underline{x} = A^{*}\underline{b}$
            \begin{align*}
                \begin{bmatrix}[ccc|c]
                    1 & 1 & i & 1-i \\
                    1 & 4 & 1 & 3-3i \\
                    -i & 1 & 3 & 1-2i
                \end{bmatrix} &\sim \begin{bmatrix}[ccc|c]
                1 & 1 & i & 1-i \\
                0 & 3 & 1 - i & 2 - 2 i \\
                0 & 1 + i & 2 & 2-i
                \end{bmatrix}
                \\
                &\sim \begin{bmatrix}[ccc|c]
                    1 & 1 & i & 1-i \\
                    0 & 1 & \frac{1-i}{3} & \frac{2-2i}{3} \\
                    0 & 0 & \frac{4}{3} & \frac{2-3i}{3}
                \end{bmatrix}
                \\
                &\sim \begin{bmatrix}[ccc|c]
                    1 & 1 & i & 1-i \\
                    0 & 1 & \frac{1-i}{3} & \frac{2-2i}{3} \\
                    0 & 0 & 1 & \frac{2-3i}{4}
                \end{bmatrix}
                \\
                &\sim \begin{bmatrix}[ccc|c]
                    1 & 1 & i & 1-i \\
                    0 & 1 & 0 & \frac{3-i}{4} \\
                    0 & 0 & 1 & \frac{2-3i}{4}
                \end{bmatrix}
                \\
                &\sim
                \begin{bmatrix}[ccc|c]
                    1 & 0 & i & \frac{1-3i}{4} \\
                    0 & 1 & 0 & \frac{3-i}{3} \\
                    0 & 0 & 1 & \frac{2-3i}{4}
                \end{bmatrix}
                \\
                &\sim \begin{bmatrix}[ccc|c]
                    1 & 0 & 0 & \frac{-2-5i}{4} \\
                    0 & 1 & 0 & \frac{3-i}{3} \\
                    0 & 0 & 1 & \frac{2-3i}{4}
                \end{bmatrix}
            \end{align*}
            Løser nå $A\underline{\hat{x}}$

            \begin{align*}
                \begin{bmatrix}
                    0 & 1 & 1 \\
                    i & i & -1 \\
                    0 & i & 1
                \end{bmatrix}\begin{bmatrix}
                \frac{-2-5i}{4} \\
                \frac{3-i}{3} \\
                \frac{2-3i}{4}
                \end{bmatrix} = \begin{bmatrix}
                \frac{17-15i}{12} \\
                \frac{13+15i}{12} \\
                \frac{10-3i}{12}
                \end{bmatrix}
            \end{align*}
        \end{punkt}
   \end{oppgave} 
    
   \begin{oppgave}
       Vi skal finne polynomer som passer til punktene \[ \begin{bmatrix}
           0 \\
           1
       \end{bmatrix}, \begin{bmatrix}
           1 \\
           2
       \end{bmatrix}, \begin{bmatrix}
           2 \\
           3
       \end{bmatrix}, \begin{bmatrix}
           3 \\
           5
       \end{bmatrix}, \begin{bmatrix}
           4 \\
           7
       \end{bmatrix} \]

       \begin{punkt}
           Det finnes et unikt fjerdegradspolynom som går gjennom alle punktene. Sett opp et ligningssystem for koeffisientene til dette polynomet, og finn koeffisientene.

           Skal finne $P(x) = a_4x^4 + a_3x^3 + a_2x^2 + a_1x + a_0$. Setter opp ligningssystem for koeffisientene $a_4, a_3, a_2, a_1$ og $a_0$.
           \begin{align*}
               \begin{cases}
                   a_0 &= 1 \\
                   a_4 + a_3 + a_2 + a_1 + a_0 &= 2 \\
                   16a_4 + 8a_3 + 4a_2 + 2a_1 + a_0 &= 3 \\
                   81a_4 + 27a_3 + 9a_2 + 3a_1 + a_0 &= 5 \\
                   256a_4 + 64a_3 + 16a_2 + 4a_1 + a_0 &= 7
               \end{cases}
           \end{align*}
           Løser ligningssystemet ved hjelp av en matrise:
           \begin{align*}
               \begin{bmatrix}[ccccc|c]
                   0 & 0 & 0 & 0 & 1 & 1 \\
                   1 & 1 & 1 & 1 & 1 & 2\\
                   16 & 8 & 4 & 2 & 1 & 3 \\
                   81 & 27 & 9 & 3 & 1& 5 \\
                   256 & 64 & 16 & 4 & 1 & 7
               \end{bmatrix} \sim \begin{bmatrix}
                   1 & 0 & 0 & 0 & 0 & -\frac{1}{12} \\
                   0 & 1 & 0 & 0 & 0 & \frac{2}{3} \\
                   0 & 0 & 1 & 0 & 0 & -\frac{17}{12} \\
                   0 & 0 & 0 & 1 & 0 & \frac{11}{6} \\
                   0 & 0 & 0 & 0 & 1 & 1
               \end{bmatrix}
           \end{align*}
           Som vil si fjerdegradspolynomet som går igjennom alle punktene er $P(x) = -\frac{1}{12}x^4 + \frac{2}{3}x^3 -\frac{17}{12}x^2 + \frac{11}{6}x + 1$
       \end{punkt}
       \begin{punkt}
           Det finnes ingen andregradspolynomer som går gjennom alle punktene. Bruk minste kvadraters metode til å finne koeffisientene til det annengradspolynomet som passer best.

           Må finne $A^{*}A\underline{x} = A^{*}\underline{b}$, der $\underline{x} = \begin{bmatrix}
               a \\
               b \\
               c
           \end{bmatrix}$ og $A = \begin{bmatrix}
           0 & 0 & 1 \\
           1 & 1 & 1 \\
           4 & 2 & 1 \\
           9 & 3 & 1 \\
           16 & 4 & 1
           \end{bmatrix}$

           \begin{align*}
               A^{*}A &= \begin{bmatrix}
                   0 & 1 & 4 & 9 & 16 \\
                   0 & 1 & 2 & 3 & 4 \\
                   1 & 1 & 1 & 1 & 1
               \end{bmatrix}\begin{bmatrix}
                   0 & 0 & 1 \\
                   1 & 1 & 1 \\
                   4 & 2 & 1 \\
                   9 & 3 & 1 \\
                   16 & 4 & 1
               \end{bmatrix}
               \\
                      &= \begin{bmatrix}
                          354 & 100 & 30 \\
                          100 & 30 & 10 \\
                          30 & 10 & 5
                      \end{bmatrix}
           \end{align*}
           
           Finner $A^{*}\underline{b}$
           \begin{align*}
               A^{*}\underline{b} &= \begin{bmatrix}
                   0 & 1 & 4 & 9 & 16 \\
                   0 & 1 & 2 & 3 & 4 \\
                   1 & 1 & 1 & 1 & 1
               \end{bmatrix}\begin{bmatrix}
                   1 \\
                   2 \\
                   3 \\
                   5 \\
                   7
               \end{bmatrix}
               \\
                                  &= \begin{bmatrix}
                                      171 \\
                                      51 \\
                                      18
                                  \end{bmatrix}
           \end{align*}
           
           Finner $A^{*}A\underline{x} = A\underline{b}$
           \begin{align*}
               \begin{bmatrix}[ccc|c]
                   354 & 100 & 30 & 171 \\
                   100 & 30 & 10 & 51 \\
                   30 & 10 & 5 & 18
               \end{bmatrix} &\sim \begin{bmatrix}[ccc|c]
               1 & \frac{100}{354} & \frac{30}{354} & \frac{171}{354} \\
               100 & 30 & 10 & 51 \\
               30 & 10 & 5 & 18
               \end{bmatrix}
               \\
               &\sim \begin{bmatrix}[ccc|c]
                   1 & \frac{100}{354} & \frac{30}{354} & \frac{171}{354} \\
                   0 & \frac{310}{177} & \frac{90}{59} & \frac{159}{59} \\
                   0 & \frac{90}{59} & \frac{145}{59} & \frac{207}{59}
               \end{bmatrix}
               \\
               &\sim \begin{bmatrix}[ccc|c]
                   1 & \frac{100}{354} & \frac{30}{354} & \frac{171}{354} \\
                   0 & 1 & \frac{27}{31} & \frac{477}{310} \\
                   0 & 1 & \frac{29}{18} & \frac{23}{10} 
               \end{bmatrix}
               \\
                &\sim \begin{bmatrix}[ccc|c]
                    1 & \frac{100}{354} & \frac{30}{354} & \frac{171}{354} \\
                    0 & 1 & \frac{27}{31} & \frac{477}{310} \\
                    0 & 0 & \frac{413}{558} & \frac{118}{155} 
               \end{bmatrix}
                \\
                &\sim \begin{bmatrix}[ccc|c]
                    1 & \frac{100}{354} & \frac{30}{354} & \frac{171}{354} \\
                    0 & 1 & \frac{27}{31} & \frac{477}{310} \\
                    0 & 0 & 1 & \frac{36}{35}
               \end{bmatrix}
                \\
                &\sim \begin{bmatrix}[ccc|c]
                    1 & \frac{100}{354} & \frac{30}{354} & \frac{171}{354} \\
                    0 & 1 & 0 & \frac{9}{14} \\
                    0 & 0 & 1 & \frac{36}{35}
               \end{bmatrix}
                \\
                &\sim \begin{bmatrix}[ccc|c]
                    1 & 0 & 0 & \frac{3}{14} \\
                    0 & 1 & 0 & \frac{9}{14} \\
                    0 & 0 & 1 & \frac{36}{35}
               \end{bmatrix}
           \end{align*}
           Andregradspolynomet som passer best er $P(x) = \frac{3}{14}x^2 + \frac{9}{14}x + \frac{36}{35}$
       \end{punkt}
   \end{oppgave}
   \begin{oppgave}
       Finn likevektsvektorene til de stokastiske matrisene:

       \begin{punkt}
            $\begin{bmatrix}
               0.8 & 0.5 \\
               0.2 & 0.5
           \end{bmatrix}$

           Likevektsvektoren til en stokastisk matrise er egenvektoren til egenverdien $\lambda = 1$. Siden jeg vet at dette er en stokastisk matrise, vet jeg også at den har en egenverdi $\lambda = 1$ og jeg kan bare finne egenvektoren:

           \begin{align*}
               \begin{bmatrix}
                   -0.2 & 0.5 \\
                   0.2 & -0.5
               \end{bmatrix} &\sim \begin{bmatrix}
               1 & -\frac{5}{2} \\
               0 & 0
               \end{bmatrix}
               \\
               \implies Null(A - I_2) &= t\cdot \begin{bmatrix}
                   \frac{5}{2} \\
                   1
               \end{bmatrix}
           \end{align*} 
           Siden likevektsvektoren også skal være en sannynlighetsvektor, må jeg finne $t$ slik at $t \cdot \begin{bmatrix}
               \frac{5}{2} \\
               1
           \end{bmatrix}$ er en sannsynlighetsvektor.
           \begin{align*}
               t(\frac{5}{2} + 1) &= 1
               \\
               \implies t &= \frac{2}{7}
               \\
               \implies q &= \begin{bmatrix}
                   \frac{5}{7} \\
                   \frac{2}{7}
               \end{bmatrix}
           \end{align*}
       \end{punkt}
       \begin{punkt}
           $
           \begin{bmatrix}
               0.7 & 0.2 & 0.2 \\
               0 & 0.2 & 0.4 \\
               0.3 & 0.6 & 0.4
           \end{bmatrix}
            $
            
           Vet igjen at denne matrisen har en egenvedri $\lambda = 1$, finner egenvektoren til denne.

           \begin{align*}
                \begin{bmatrix}
                    -0.3 & 0.2 & 0.2 \\
                   0 & -0.8 & 0.4 \\
                   0.3 & 0.6 & -0.6
                \end{bmatrix} &\sim \begin{bmatrix}
           -0.3 & 0.3 & 0.2 \\
           0 & -0.8 & 0.4 \\
           0 & 0.8 & -0.4
           \end{bmatrix}
           \\            
           &\sim \begin{bmatrix}
               -0.3 & 0.2 & 0.2 \\
               0 & 0 & 0  \\
               0 & 1 & -0.5
           \end{bmatrix}
           \\
           &\sim \begin{bmatrix}
               1 & -2 & 0 \\
               0 & 0 & 0 \\
               0 & 1 & -0.5
           \end{bmatrix}
           \\
               \implies Null(A - I_{3}) &= t\cdot \begin{bmatrix}
                   2 \\
                   0.5 \\
                   1
               \end{bmatrix}
           \end{align*}
           Finner igjen en $t$ slik at elementene i $q$ summerer opp til $1$.
           \begin{align*}
               t(2 + 0.5 + 1) &= 1
                \\
               \implies t &= \frac{2}{7}
               \\
               \implies q = \begin{bmatrix}
                   \frac{4}{7} \\
                   \frac{1}{7} \\
                   \frac{2}{7}
               \end{bmatrix}
           \end{align*}
       \end{punkt}
   \end{oppgave}
   \begin{oppgave}
       Er følgende stokastiske matriser regulære?
       \begin{punkt}
           $P = \begin{bmatrix}
               0.2 & 1 \\
               0.8 & 0
           \end{bmatrix}$

           Må finne en $k \geq 1$ slik at alle elementene i $M^k \geq 0$.

           \begin{align*}
               P^2 &= \begin{bmatrix}
                   0.2 & 1 \\
                   0.8 & 0
               \end{bmatrix}\begin{bmatrix}
                   0.2 & 1 \\
                   0.8 & 0
               \end{bmatrix}
               \\
                   &= \begin{bmatrix}
                       \frac{21}{25} & \frac{1}{5} \\
                       \frac{4}{25} & \frac{4}{5}
                   \end{bmatrix}
           \end{align*}
           Ja, denne matrisen er regulær.
       \end{punkt}
       \begin{punkt}
           $Q = \begin{bmatrix}
               1 & 0.2 \\
               0 & 0.8
           \end{bmatrix}$

           Ser med en gang at denne ikke er regulær, da den har 0 i nedre hjørne. Hver gang en multipliserer denne matrisen med seg selv vil det alltid bli 0 i dette hjørnet. 
       \end{punkt}
   \end{oppgave}

   \begin{oppgave}
       Temperaturen i Bymarka i løpet av vintersesongen kan enten være over, lik, eller under $0^{\circ}$ Celsius. Trondheims skiklubb observerte de følgende svingningene i temperatur fra den ene dagen til den neste:

       \begin{itemize}
           \item Når temperaturen har vært over $0$, er det $70\%$ sannsynlighet for at den vil være over og $10\%$ sannsynlighet for at den vil være under $0^{\circ}$ neste dag.
           \item Når temperaturen har vært lik $0^{\circ}$, er det $10\%$ sannsynlighet for at den vil være over og $10\%$ sannsynlighet for at den vil være under $0^{\circ}$ neste dag.
           \item Når temperaturen har vært under $0^{\circ}$, er det $10\%$ sannsynlighet for at den vil være over og $70\%$ sannsynlighet for at den vil være under $0^{\circ}$ neste dag.
       \end{itemize}
       Etter mange dager med dette mønsteret i vinter, for hvilken temperatur bør en skiløper forberede sine ski? (Gi sannsynlighetene for de tre mulige temperaturene.)

       
       Setter opp dette som en tabell først:
       \begin{table}[htpb]
           \centering
           \begin{tabular}{c|c|c|c|c}
               Fra & Over & Under & Lik & Til \\
               \hline
                   & $0.7$ & $0.1$ & $0.1$ & Over \\
                   & $0.1$ & $0.7$ & $0.1$ & Under \\
                   & $0.2$ & $0.2$ & $0.8$ & Lik
           \end{tabular}
       \end{table}
       
       Kan nå sette det opp som en matrise $M$:
       \begin{align*}
           M &= \begin{bmatrix}
               0.7 & 0.1 & 0.1 \\
               0.1 & 0.7 & 0.1 \\
               0.2 & 0.2 & 0.8
           \end{bmatrix}
       \end{align*}
       
       For å finne de tre sannsynlighetene må jeg finne likevektsvektoren. Gjør dette ved å finne egenvektor til $\lambda = 1$:
       \begin{align*}
           \begin{bmatrix}
           -0.3 & 0.1 & 0.1 \\
           0.1 & -0.3 & 0.1 \\
           0.2 & 0.2 & -0.2
       \end{bmatrix} &\sim \begin{bmatrix}
       -3 & 1 & 1 \\
       1 & -3 & 1 \\
       2 & 2 & -2
       \end{bmatrix}
       \\
       &\sim \begin{bmatrix}
           -3 & 1 & 1 \\
           1 & -3 & 1 \\
           1 & 1 & -1
       \end{bmatrix}
       \\
       &\sim \begin{bmatrix}
           0 & 0 & 0 \\
           0 & 1 & -\frac{1}{2} \\
           1 & 0 & -\frac{1}{2}
       \end{bmatrix}
       \\
           \implies q &= t \cdot \begin{bmatrix}
               \frac{1}{2} \\
               \frac{1}{2} \\
               \frac{1}{4}
           \end{bmatrix}
       \end{align*}
       Setter jeg $t = \frac{1}{2}$, får jeg $q = \begin{bmatrix}
           \frac{1}{4} \\
           \frac{1}{4} \\
           \frac{1}{2}
       \end{bmatrix}$

       Skiløperen må beregne at temperaturen er lik $0^{\circ}$
   \end{oppgave}
   \begin{oppgave}
       Vis at en regulær stokastisk $2$x$2$ matrise \[ M = \begin{bmatrix}
           1-a & b \\
           a & 1-b
       \end{bmatrix} \; \; \text{med} \; \; 0 < a,b < 1\] har en unik likevektsvektor.

       Finner egenverdiene til matrisen $M$:
       \begin{align*}
           det(M - \lambda I_2) &= (1-a-\lambda)(1-b-\lambda)-ab
           \\
                                &= \lambda^2 - 2\lambda + a\lambda + b\lambda - a - b + 1 + ab - ab
                                \\
                                &= \lambda^2 + \lambda(a+b-2) - a - b + 1
                                \\
                                \\
           \implies \lambda &= \frac{-a-b+2 \pm \sqrt{(a+b-2)^2 - 4 \cdot (-a-b+1)}}{2}
           \\
                            &= \frac{-a-b+2 \pm \sqrt{a^2 + 2ab + b^2}}{2} 
                            \\
                            &= \frac{-a-b+2 \pm \sqrt{(a+b)^2}}{2}
                            \\
           \implies \lambda_1 &= \frac{-a-b+2-a-b}{2} = -a-b+1, \; \lambda_2 = \frac{-a-b+2+a+b}{2} = 1
       \end{align*}
       Ser at matrisen har to egenverier, der den ene er $1$. Denne egenverdien gir likevektsvetoren:
       \begin{align*}
           \begin{bmatrix}
               1-a-1 & b \\
               a & 1-b-1
           \end{bmatrix} &\sim \begin{bmatrix}
           -a & b \\
            a & -b
           \end{bmatrix}
           \\
           &\sim \begin{bmatrix}
               1 & -\frac{b}{a}\\
               0 & 0
           \end{bmatrix}
           \\
           \implies q &= t\cdot \begin{bmatrix}
               \frac{b}{a} \\
               1
           \end{bmatrix}
       \end{align*}
       For at $q$ faktisk skal være en likevektsvektor må elementene i vektoren summere til $1$:
       \begin{align*}
           t(\frac{b}{a} + 1) &= 1 
           \\
           \implies t &= \frac{1}{\frac{b}{a} + 1}
           \\
           \implies q &= \begin{bmatrix}
               \frac{b}{b+a} \\
               \frac{1}{\frac{b}{a} + 1}
           \end{bmatrix}
       \end{align*}
       
   \end{oppgave}
\end{document}
